\documentclass{article}
\usepackage[russian]{babel}
\usepackage[utf8]{inputenc}

\begin{document}

\section{Общее знакомство с Unix-подобными ОС на примере Linux}

    \begin{itemize}
        \item Особенности ФС, отличия от ФС Windows: единое дерево каталогов, отсутствие диска C:, наличие rwx-битов, файлы и ФС особого вида (файлы устройств, символические ссылки, ФС /proc, /sys и /dev)
        \item Краткий обзор системных каталогов (на сайте GNU есть подробное описание)
        \item Знакомство с bash: навигация по ФС, создание и удаление файлов и папок, запуск приложений, передача аргументов командной строки
        \item Обзор часто используемых Unix-утилит: ls, cat, grep, ps, man, chmod и т.д.
        \item bash-скрипты: переменные, условия, циклы, конструкция "\$()", специальные переменные для аргументов (\$0, \$1, ..., \$@)
        \item Пакеты и пакетные менеджеры, дистрибутивы
    \end{itemize}

    ДЗ:
    \begin{itemize}
        \item Написать утилиту pcat для bash: она принимает список файлов, для каждого файла выводит его имя, затем печатает содержимое файла. Обрабатывать ошибки не нужно
        \item Vimtutor/Emacs
    \end{itemize}

\section{Системы контроля версий}

    \begin{itemize}
        \item Обзор: SVN, hg, git
        \item Git: clone, pull, commit, push, merge, rebase, cherry-pick
    \end{itemize}

    ДЗ: git clone, test commit

\section{Программы на языке Си}

    \begin{itemize}
        \item Компилятор GCC, пример отладки с помощью GDB (псевдографический режим включается командой "+")
        \item Особенности языка Си (студенты уже знают C++98)
        \begin{itemize}
            \item Структуры, битовые поля
            \item typedef
            \item Указатели на функции
            \item Функции с переменным числом аргументов
            \item Стандартные функции ввода/вывода
            \item Работа с динамической памятью
            \item Статические переменные и их область видимости
            \item     preprocessing: define, ifdef...
            \item    extern, volatile, auto
            \item   Глобальные переменные, их нехорошесть
        \end{itemize}
        \item Краткое описание некоторых стандартных заголовочных файлов и часто используемых заголовков из POSIX
        \item Общее представление о системных вызовах
        \item Этапы получения бинарника из исходников: компиляция, линковка
        \item Пример программы из нескольких исходных файлов
        \item Makefile и CMake
        \item Знакомство с binutils
        \item Динамические библиотеки: зачем нужны, как создавать, как использовать
        \item getopt()
    \end{itemize}

    Опционально:
        \begin{itemize}
            \item Функции dlopen(), dlinfo(), dlsym() и т.д.
            \item Пример использования Valgrind для поиска утечек памяти
        \end{itemize}

    ДЗ:
    \begin{itemize}
        \item Написать динамическую библиотеку, экспортирующую функцию int sum(int cnt, ...)
        \item Написать программу, использующую эту функцию, и слинковать её с этой библиотекой
        \item Написать Makefile или CMakeLists.txt для этого
    \end{itemize}

\section{Ввод-вывод в UNIX-подобных системах}

\begin{itemize}
    \item Функции fopen(), fclose(), freopen() и т.д. из стандартной библиотеки Си
    \item Файловые дескрипторы, fdopen()
    \item Системные вызовы open(), read(), write(), ioctl(), fcntl(), fstat(), flock(), dup() и dup2()
    \item Функции для работы с каталогами: opendir(), readdir(), scandir(), closedir() и т.д.
    \item Terminfo
\end{itemize}

\section{Ассемблер}

    \begin{itemize}
        \item Регистровая память: зачем нужна, особенности работы, ключевое слово register в Си
        \item Оперативная память: структура адресного пространства процесса (какие секции лежат в адресном пространстве)
        \item Стек и куча: назначение, отличия между ними, как с ними работать
        \item Cdecl: зачем нужно, как устроено
        \item GNU Assembler:
        \begin{itemize}
            \item Секции
            \item Метки
            \item Выделение памяти в секциях .data, .rodata, .bss
            \item Часто используемые инструкции: nop, mov, арифметические, логические, cmp, test, условный и безусловный переходы, push, pop, call, ret
            \item Пример программы на Assembler
        \end{itemize}
        \item Компиляция маленькой программы на Си в ассемблерный код: gcc --S
    \end{itemize}

    Опционально:
    \begin{itemize}
        \item Работа с системными вызовами на Assembler в Linux
        \item как обратиться к ядру, минуя libc
        \item Readelf
    \end{itemize}

    ДЗ:
    \begin{itemize}
        \item Написать рекурсивную функцию factorial на ассемблере, которая приниамет 4-байтовое число n и возвращает 4-байтовое число n!
        \item Написать программу на ассемблере, которая читает натуральное число из стандартного потока ввода и выводит его факториал в стандартный поток вывода. Некорректный ввод можно не обрабатывать. Для ввода-вывода следует использовать обёртки вокруг printf() и scanf(), написанные на Си (напр., функции int read\_int() и void print\_int(int x))
        \item Makefile или CMakeLists.txt (https://cmake.org/Wiki/CMake/Assembler)
    \end{itemize}

\section{Threads}

    \begin{itemize}
        \item Концепция многопоточного процесса, отличия потоков от процессов
        \item Работа с потоками: библиотека pthread, функции для работы с потоками
        \item Опционально: системный вызов clone()
    \end{itemize}

\section{Сигналы}

    \begin{itemize}
        \item Концепция сигналов, назначение, преимущества и недостатки
        \item Обработчики сигналов, функция sigaction()
        \item ptrace
    \end{itemize}

\section{IPC}

    \begin{itemize}
        \item Разделяемая память, функции для работы с нею
        \item Очереди сообщений
    \end{itemize}

\section{Механизмы синхронизации процессов (здесь должно быть много примеров и кода)}

    \begin{itemize}
        \item Синхронизация потоков
        \item Мьютексы: пример race condition, способ этого избежать, функции для работы с мьютексами
        \item Семафоры: задача производителя и потребителя, использование семафоров для решения этой задачи, функции для работы с семафорами
        \item Разница между двоичным семафором и мьютексом
    \end{itemize}

    Опционально: Исторический экскурс о важности правильного проектирования многопоточных систем: Therac-25
       
    ДЗ:
    \begin{itemize}
        \item Реализуйте эмуляцию сборочного цеха. Есть три автомата: первый каждую секунду производит деталь A, второй каждые две секунды
                производит деталь B, а третий ждёт, пока появятся доступные детали и за три секунды из двух деталей A и одной детали B собирает
                деталь C. При этом автоматы пишут сообщения о своей работе в журнал (stdout): при появлении очередной детали автоматы делают
                соответствующую запись. При сигнале SIGUSR1 в журнале появляется запись о количестве деталей каждого вида.
       
        \item Добавьте механизм синхронизации, не допускающий перемешивания строк в журнале.
    \end{itemize}

\section{FIFO, pipe, sockets (PF\_UNIX)}

    \begin{itemize}
        \item Именованные каналы: назначение, принцип действия
        \item Сокеты: назначение, создание, настройка, использование
        \item Взгляд на сетевое взаимодействие со стороны сервера и клиента (красивая картинка с диаграммой системных вызовов)
        \item TCP, UDP: отличия от TCP, область применения, функции для работы с UDP-сокетами
       
        ДЗ: Написать echo-сервер на TCP-сокетах: ожидает одно соединение, затем, пока приходят сообщения от клиента, отправлять их обратно.
            Уметь поддерживать несколько соединений одновременно не нужно.
    \end{itemize}

\section{Многопроцессность, неименованные каналы}

    \begin{itemize}
        \item Концепция процесса в многопользовательской системе, методы реализации
        \item Создание нового процесса: системный вызов fork()
        \item Взаимодействие процессов через неименованные каналы, функция pipe()
        \item Замещение тела процесса, системный вызов execve()
        \item Опционально: алгоритмы работы планировщика процессов
    \end{itemize}
        
     ДЗ:
         Написать программу, которая создаёт два дочерних процесса. Каждый из этих процессов использует свой неименованный канал.
         Первый дочерний процесс пишет в канал нечётные числа от 1 до 99, второй - чётные от 2 до 100. Задача родительского
         процесса - читать эти числа поочерёдно и выводить их на стандартный поток вывода.

\section{Доп. семинар по сокетам}

    \begin{itemize}
        \item Что такое DNS-сервер, для чего нужен, функции для работы с DNS
        \item мультиплексирование, пример трёх абонентов на одном TCP-соединении (fork() после accept())
        \item Почему отдельный поток для каждого сокета - это плохо, как решить эту проблему, функции poll() и select()
        \item Защита передачи данных
        \item Недостатки TCP: возможность перехвата и подмены пакетов
    \end{itemize}

    Опционально:
    \begin{itemize}
        \item Краткое введение в криптографию: симметричное и асимметричное шифрование, цифровая подпись, описание алгоритмов RSA и Diffie-Hellman
        \item Принцип работы протоколов SSL/TLS
    \end{itemize}

\section{Доп. семинар по UNIX}

    \begin{itemize}
        \item Программы для работы с текстом: sed, awk, grep, diff
        \item Программы для работы с архивами: tar, gzip, bzip2, xzip, ...
        \item Программы для администрирования: ifconfig, htop/top, du/ncdu, df, cron
        \item Программы для работы с сетью: ping, nslookup, nmap, wget/curl
        \item Прочее: bc, xargs, apropos, dd
        \item sysinfo, ulimits
    \end{itemize}

\section{Доп. семинар по ncurses, readline}

    \begin{itemize}
        \item Сессии, группы процессов, владение терминалом
        \item Режимы терминала
        \item gettext, локализация, LC\_ALL, LC\_LANG, ...
    \end{itemize}

\end{document}
