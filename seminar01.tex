\documentclass{beamer}
\usetheme{Warsaw}
\usepackage[utf8]{inputenc}
\usepackage[russian]{babel}
\usepackage{default}


\usepackage{fancyvrb}

\usepackage{tikz}
\usetikzlibrary{positioning,arrows,shapes,shadows,calc}

\usepackage{graphicx}

\usepackage{hyperref}

\title[АКОС --- семинар 1]{Общее знакомство с Unix-подобными ОС на примере Linux}
\date{}
	
\tikzstyle{dir} = [
ellipse,
thick,
minimum size=1cm,
draw=red!50!black!50,
top color=white,
bottom color=red!50!black!20,
drop shadow
]
	
\tikzstyle{file} = [
	rounded rectangle,
	thick,
	minimum size=1cm,
	draw=blue!50!black!50,
	top color=white,
	bottom color=blue!50!black!20,
	drop shadow
]

\tikzstyle{blkdev} = [
	diamond,
	thick,
	minimum size=1cm,
	draw=yellow!50!black!50,
	top color=white,
	bottom color=yellow!50!black!20,
	drop shadow
]

\begin{document}

\begin{frame}
	\maketitle
\end{frame}

\begin{frame}{Содержание}
	\tableofcontents
\end{frame}

\section{Особенности файловой системы}
\subsection{Отличия от ФС Windows}
\begin{frame}{Особенности ФС Unix}
	В отличие от ФС Windows, каталоги в ФС Unix образуют единое дерево: нет привычных дисков C:, D: и так далее
	
	\begin{tikzpicture}[
		align=center,
		thick,
		node distance=1.5cm,
		text height=1.5ex,
		text depth=.25ex,
		auto
	]
	
		\node[dir] (etc) {etc};
		\node[dir,right of=etc] (var) {var};
		\node[dir,right of=var] (usr) {usr};
		\node[dir,right of=usr] (bin) {bin};
		\node[dir,right of=bin] (dev) {dev};
		\node[dir,right of=dev] (home) {home};
		
		\node[dir,above of=var] (root) at ($(bin)!0.5!(usr)$) {/};

		\node[file,below of=etc] (passwd) {passwd};
		\node[dir,below of=usr] (lib) {lib};
		\node[dir,below of=var] (usrbin) {bin};
		\node[file,below of=bin] (bash) {bash};
		\node[blkdev,below of=dev] (sda) {sda};
		\node[dir,below of=home] (alex) {alex};
		\node[dir,right of=alex] (bob) {bob};

		\path[->] (root) edge (etc) edge (var) edge (usr) edge (bin) edge (dev) edge (home);
		\path[->] (etc) edge (passwd);
		\path[->] (usr) edge (usrbin) edge (lib);
		\path[->] (bin) edge (bash);
		\path[->] (dev) edge (sda);
		\path[->] (home) edge (alex) edge (bob);
	\end{tikzpicture}
\end{frame}

\begin{frame}
	\begin{itemize}
		\item{При подключении флешки или другого съёмного носителя файловую систему, содержащуюся на этом носителе, необходимо \emph{смонтировать} в специальный каталог}\pause
		\item{Для этого существует специальная программа \texttt{mount}: например, если на флешке есть файл \texttt{smile.png}, и Вы хотите смонтировать флешку в каталог \texttt{/mnt/flash}, то после монтирования этот файл станет доступен по пути \texttt{/mnt/flash/smile.png}}\pause
		\item{Большинство дистрибутивов Linux поддерживают автоматическое монтирование съёмных носителей}
	\end{itemize}
\end{frame}

\begin{frame}
	\begin{itemize}
		\item{Перед тем, как отсоединить флешку от компьютера, её необходимо \emph{отмонтировать} (в Windows это называется <<Безопасное извлечение устройства>>)}\pause
		\item{Для этого используется специальная программа \texttt{umount} (буква n потерялась в силу исторических причин)}
	\end{itemize}
\end{frame}

\subsection{Права доступа}

\begin{frame}{Права доступа}
	В файловой системе UNIX с каждым файлом, помимо имени, связан набор \emph{атрибутов}:\pause
	\begin{itemize}
%		\item{Имя файла}\pause
		\item{Каталог, в котором файл лежит}\pause
		\item{Владелец файла}\pause
		\item{Время создания, модификации, обращения к файлу}\pause
		\item{Права доступа к файлу}\pause
		\item{И так далее}
	\end{itemize}
\end{frame}

\begin{frame}[fragile]
	\begin{itemize}
		\item{Права доступа к файлу или каталогу задаются специальным 12-битным кодом режима доступа}\pause
		\item{Этот код состоит из 3 битов, влияющих на запуск исполняемых файлов, и 3 трёхбитных полей, сокращённо обозначаемых rwx-битами}\pause
		\item{Обычно эти биты представляются так:}\pause
	\end{itemize}
	\begin{Verbatim}[commandchars=\\\{\},codes={\catcode`$=3\catcode`^=7\catcode`_=8}]
итого 56K
{\color{red}-rwxr-xr-x} 1 alex students  11K июл  8 17:29 bar
{\color{red}-rw-r--r--} 1 alex students  13K июл  8 17:29 CMakeCache.txt
{\color{red}drwxr-xr-x} 6 alex students 4,0K июл  8 17:29 CMakeFiles/
{\color{red}-rwxr-xr-x} 1 alex students  11K июл  8 17:29 foo
{\color{red}-rw-r--r--} 1 alex students 5,9K июл  8 17:29 Makefile
	\end{Verbatim}
\end{frame}

\begin{frame}[fragile]
	\begin{Verbatim}[commandchars=\\\{\},codes={\catcode`$=3\catcode`^=7\catcode`_=8}]
{\color{red}-rwxr-xr-x} 1 alex students  11K июл  8 17:29 bar
	\end{Verbatim}
	\begin{itemize}
		\item{Первый символ отвечает за тип файла (в данном листинге есть только простые файлы и один каталог)}\pause
		\item{Затем идут три rwx-группы: первая из них отвечает за права владельца файла (\texttt{alex}), вторая --- за права группы-владельца (\texttt{students}), а третья --- за права всех остальных пользователей}\pause
		\item{Бит \texttt{r} обозначает, что этот файл можно читать}\pause
		\item{Бит \texttt{w} обозначает, что в этот файл можно писать}\pause
		\item{Бит \texttt{x} обозначает, что этот файл исполняемый}
	\end{itemize}
\end{frame}

\begin{frame}[fragile]
	\begin{Verbatim}[commandchars=\\\{\},codes={\catcode`$=3\catcode`^=7\catcode`_=8}]
{\color{red}drwxr-xr-x} 6 alex students 4,0K июл  8 17:29 CMakeFiles/
	\end{Verbatim}
	\begin{itemize}
		\item{В случае каталога эти биты имеют немного другой смысл. Так, если бит \texttt{r} установлен, то Вы можете получить список файлов, лежащих в этом каталоге, бит \texttt{w} позволяет создавать и удалять файлы, а от бита \texttt{x} зависит, сможете ли Вы узнать список атрибутов, связанных с файлом из этого каталога}\pause
		\item{Забавная ситуация возникает, когда у каталога установлены права \texttt{-wx}. В этом случае невозможно узнать напрямую, какие файлы лежат в каталоге, зато возможно создание, удаление, изменение файлов и подкаталогов (на практике чаще всего используют набор прав 7 (\texttt{rwx}), 5 (\texttt{r-x}) и 0 (\texttt{-{}-{}-}))}
	\end{itemize}
\end{frame}

\begin{frame}
	\begin{itemize}
		\item{Помимо этих 9 битов, существуют биты SUID, SGID и sticky}\pause
		\item{Если у каталога стоит бит sticky, то никто, кроме владельца каталога и суперпользователя, не сможет удалить файлы из этого каталога (применительно к другим типам файлов этот бит бесполезен и игнорируется ОС)}\pause
		\item{Биты SUID и SGID влияют на то, от чьего имени будет запущена программа}
	\end{itemize}
\end{frame}

\subsection{Файлы особого вида}
\begin{frame}{Файлы особого вида}
	В Linux, помимо обычных файлов и каталогов, существуют особенные файлы, которые не являются файлами в привычном смысле. Это могут быть устройства ввода-вывода или различные системные структуры, с которыми удобно взаимодействовать так, будто это файлы. К ним относятся: \pause
	\begin{itemize}
		\item{\texttt{l} --- символические ссылки (что-то вроде ярлыков в Windows);}\pause
		\item{\texttt{b} --- блочные устройства (например, жесткие диски);}\pause
		\item{\texttt{c} --- символьные устройства (мышь, клавиатура);}\pause
		\item{\texttt{p} --- именованные каналы (будут подробно рассмотрены позднее);}\pause
		\item{\texttt{s} --- UNIX-сокеты}
	\end{itemize}
\end{frame}

\section{Обзор системных каталогов UNIX}
\begin{frame}{Обзор системных каталогов UNIX}
	В большинстве UNIX-подобных систем используется схожий набор системных файлов. Поэтому структура ФС у них очень похожа. Для её описания используется т.н. FHS --- стандарт иерархии ФС. Вот основные каталоги, используемые в UNIX-системах:\pause
	\begin{itemize}
		\item{\texttt{/bin} --- основные утилиты, например cat, ls, cp}\pause
		\item{\texttt{/boot} --- загрузочные файлы (настройки загрузчика, ядро и т.д.); часто выносятся на отдельный раздел диска}\pause
		\item{\texttt{/dev} --- основные файлы устройств}\pause
		\item{\texttt{/etc} --- общесистемные конфигурационные файлы}\pause
		\item{\texttt{/home} --- содержит домашние каталоги пользователей}\pause
		\item{\texttt{/lib} --- основные библиотеки}
	\end{itemize}
\end{frame}

\begin{frame}
	\begin{itemize}
		\item{\texttt{/proc} --- виртуальная файловая система, представляющая состояние ядра и запущенных процессов в виде файлов}\pause
		\item{\texttt{/root} --- домашний каталог пользователя \texttt{root}}\pause
		\item{\texttt{/sbin} --- основные системные программы для администрирования}\pause
		\item{\texttt{/tmp} --- временные файлы, которые будут удалены при перезагрузке}\pause
		\item{\texttt{/usr} --- пользовательские приложения и дополнительные утилиты (в подкаталогах \texttt{/usr/bin}, \texttt{/usr/sbin}, \texttt{/usr/lib} и других)}\pause
		\item{\texttt{/var} --- изменяемые файлы (логи, кэш приложений, почтовые ящики и др.)}
	\end{itemize}
\end{frame}

\section{Знакомство с bash}

\begin{frame}{Знакомство с bash}
		Одним из основных способов взаимодействия с UNIX-системами являются командные оболочки UNIX. В некотором смысле это аналоги командной строки Windows. Наиболее часто используемая командная оболочка --- это \textbf{B}ourne \textbf{a}gain \textbf{sh}ell, она же bash.
		
		\begin{center}
			\includegraphics[width=5cm]{bash}
			
			{\color{gray}\tiny Источник картинки: \color{blue}\url{https://habrahabr.ru/post/127084/}}
		\end{center}
\end{frame}

\subsection{Навигация по ФС}

\begin{frame}{Навигация по файловой системе}
	
\end{frame}

\end{document}
