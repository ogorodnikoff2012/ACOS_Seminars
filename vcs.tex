\PassOptionsToPackage{unicode=true}{hyperref}

% print mode
%\documentclass[handout]{beamer}
% presentation mode
\documentclass{beamer}

\usetheme{Montpellier}
\usepackage[T2A]{fontenc}
\usepackage[utf8x]{inputenc}
\usepackage[russian]{babel}
\usepackage{default}

\usepackage{minted}

\usepackage{fancyvrb}

\usepackage{tikz}
\usetikzlibrary{positioning,arrows,shapes,shadows,calc}

\usepackage{graphicx}


\tikzstyle{dir} = [
ellipse,
thick,
minimum size=1cm,
draw=red!50!black!50,
top color=white,
bottom color=red!50!black!20,
drop shadow
]

\tikzstyle{workdir} = [
ellipse,
thick,
minimum size=1cm,
draw=red!50!black!70,
top color=white,
bottom color=red!50!black!50,
drop shadow
]

\tikzstyle{file} = [
rounded rectangle,
thick,
minimum size=1cm,
draw=blue!50!black!50,
top color=white,
bottom color=blue!50!black!20,
drop shadow
]

\tikzstyle{blkdev} = [
diamond,
thick,
minimum size=1cm,
draw=yellow!50!black!50,
top color=white,
bottom color=yellow!50!black!20,
drop shadow
]

\definecolor{listing}{rgb}{0.95,0.95,0.95}

\title{Системы контроля версий}
\date{}

\begin{document}
\begin{frame}
	\maketitle
\end{frame}

\begin{frame}[allowframebreaks]{Содержание}
	\tableofcontents
\end{frame}

\section{Знакомство с системами контроля версий}

\begin{frame}{Предпосылки создания СКВ}
	\begin{itemize}
		\item{При разработке различных проектов очень часто возникает необходимость как-то хранить предыдущие версии проекта: например, в текущей версии есть ошибка, и нужно вернуться к той реализации, которая была в прошлых версиях. Кроме того, наличие резервных копий помогает избежать многих неприятностей}\pause
		\item{Самый примитивный способ --- это просто делать архивы с исходниками. Проблемы начинаются при совместной работе. Совмещать разные версии большого проекта, тем более вручную, тем более в том случае, когда несколько человек внесли правки в один и тот же файл --- весьма неприятная задача}\pause
		\item{Для автоматизации решения этих задач были созданы \emph{системы контроля версий} (version control system)}
	\end{itemize}	
\end{frame}

\begin{frame}{Современные СКВ}
	\begin{itemize}
		\item{В данный момент существуют десятки различных СКВ. Наиболее популярной на данный момент является система git, за ней с огромным отрывом идёт Subversion, затем Mercurial и другие}\pause 
		\item{Системы Subversion, CVS и другие называются \emph{централизованными}: есть центральный сервер, который хранит информацию обо всех версиях, и есть множество пользователей, у каждого из которых своя рабочая копия нужных ему файлов. Работать с такими системами гораздо удобнее, чем с локальными СКВ. Однако в случае недоступности центрального сервера работа с проектом становится невозможной. Кроме того, полный архив всех версий есть только в одном экземпляре.}
	\end{itemize}
\end{frame}

\begin{frame}
	Эти недостатки решаются в \emph{распределённых} СКВ, к которым относятся git, Mercurial, Bazaar и другие. В отличие от ЦСКВ, эти системы хранят на пользовательских машинах весь репозиторий целиком. Это позволяет работать с несколькими удалёнными репозиториями (скажем, хранить разные ветки одного и того же проекта на разных серверах) или же обходиться без удалённых репозиториев вообще (такая возможность есть, например, в git)
\end{frame}

\subsection{Знакомство с Git}

\begin{frame}{Знакомство с Git}
	\begin{itemize}
		\item{Как уже было сказано ранее, git --- это распределённая система контроля версий. Эта система была создана в 2005 году сообществом разработчиков ядра Linux на смену пропиетарной BitKeeper. Git --- это очень гибкий инструмент. У этого проекта есть подробное руководство на русском языке: {\color{blue} \url{https://git-scm.com/book/ru/v1}}. В данной презентации дано краткое описание часто используемых команд}\pause
		\item{Для начала работы используются две команды: \texttt{git init <dir>} и \texttt{git clone <url>}. Первая позволяет создать локальный репозиторий из уже существующего каталога с файлами. Вторая же предназначена для создания локальной копии уже существующего репозитория.}
	\end{itemize}
\end{frame}

\begin{frame}
	\begin{itemize}
		\item{Для того, чтобы загрузить последнюю версию проекта, нужно выполнить команду \texttt{git pull}. Вообще, во избежание конфликтов версий при командной работе над проектом следует делать это всякий раз перед началом работы.}\pause
		\item{Чтобы загрузить изменения в репозиторий, нужно выполнить несколько операций. Во-первых, git даёт возможность пользователю хранить <<лишние>> файлы в каталоге с проектом (например, конфигурационные файлы IDE или логи). Кроме того, какие-то важные файлы могли быть случайно удалены, переименованы или изменены.}\pause
		\item{Поэтому про \emph{все} полезные изменения нужно сообщить git с помощью команды \texttt{git add <file1> <file2> ...} (вместо файлов можно указывать каталоги; если нужно добавить все изменения, можно применить команду \texttt{git add .}, находясь в корневом каталоге проекта)}
	\end{itemize}
\end{frame}

\begin{frame}
	\begin{itemize}
		\item{Чтобы увидеть, какие файлы нуждаются в индексировании и какие изменились по сравнению с предыдущей версией, или \emph{коммитом}, как это принято называть в git, воспользуйтесь командой \texttt{git status}}\pause
		\item{Чтобы увидеть разницу между теми файлами, которые реально есть в рабочем каталоге, и файлами в индексе git, выполните команду \texttt{git diff}. Добавление ключа \texttt{-{}-staged} или \texttt{-{}-cached} позволит увидеть разницу между проиндексированными изменениями и последним коммитом}\pause
		\item{Наконец, когда все нужные изменения проиндексированы, нужно все эти изменения зафиксировать. Делается это с помощью команды \texttt{git commit}. При этом откроется текстовый редактор, в котором нужно написать некоторое пояснение к коммиту. Чтобы избедать этого, можно выполнить коммит с ключом \texttt{-m}: \texttt{git commit -m "Msg"}}
	\end{itemize}
\end{frame}

\begin{frame}
	\begin{itemize}
		\item{Для синхронизации с удалённым репозиторием нужно воспользоваться командой \texttt{git push}. Будьте готовы к тому, что у Вас попросят пароль, если, конечно, не настроена аутентификация по ключу}\pause
		
	\end{itemize}
\end{frame}

\end{document}
