\documentclass{article}
\usepackage[T2A]{fontenc}
\usepackage[utf8x]{inputenc}
\usepackage[russian]{babel}
\usepackage{default}
\usepackage{indentfirst}
\usepackage{hyperref}
\usepackage{color}

\title{Системы контроля версий}
\date{}
\begin{document}

\maketitle
\tableofcontents

\section{Знакомство с системами контроля версий}
\subsection{}
При разработке различных проектов очень часто возникает необходимость как-то хранить предыдущие версии проекта: например, в текущей версии есть ошибка, и нужно вернуться к той реализации, которая была в прошлых версиях. Кроме того, наличие резервных копий помогает избежать многих неприятностей.

Самый примитивный способ --- это просто делать архивы с исходниками. Проблемы начинаются при совместной работе. Совмещать разные версии большого проекта, тем более вручную, тем более в том случае, когда несколько человек внесли правки в один и тот же файл --- весьма неприятная задача.

Для автоматизации решения этих задач были созданы \emph{системы контроля версий} (version control system).

\subsection{}
В данный момент существуют десятки различных СКВ. Наиболее популярной на данный момент является система git, за ней с огромным отрывом идёт Subversion, затем Mercurial и другие.

Системы Subversion, CVS и другие называются \emph{централизованными}: есть центральный сервер, который хранит информацию обо всех версиях, и есть множество пользователей, у каждого из которых своя рабочая копия нужных ему файлов. Работать с такими системами гораздо удобнее, чем с локальными СКВ. Однако в случае недоступности центрального сервера работа с проектом становится невозможной. Кроме того, полный архив всех версий есть только в одном экземпляре.

Эти недостатки решаются в \emph{распределённых} СКВ, к которым относятся git, Mercurial, Bazaar и другие. В отличие от ЦСКВ, эти системы хранят на пользовательских машинах весь репозиторий целиком. Это позволяет работать с несколькими удалёнными репозиториями (скажем, хранить разные ветки одного и того же проекта на разных серверах) или же обходиться без удалённых репозиториев вообще (такая возможность есть, например, в git).

\section{Знакомство с git}

\subsection{}
Как уже было сказано ранее, git --- это распределённая система контроля версий. Эта система была создана в 2005 году сообществом разработчиков ядра Linux на смену пропиетарной BitKeeper. Git --- это очень гибкий инструмент. У этого проекта есть подробное руководство на русском языке: {\color{blue} \url{https://git-scm.com/book/ru/v1}}. В данной презентации дано краткое описание часто используемых команд.

Для начала работы используются две команды: \texttt{git init <dir>} и \texttt{git clone <url>}. Первая позволяет создать локальный репозиторий из уже существующего каталога с файлами. Вторая же предназначена для создания локальной копии уже существующего репозитория.

\subsection{}
В основе репозиториев git лежат \emph{коммиты} (в других СКВ они называются слепками, версиями, ревизиями). Коммит --- это, по большому счёту, слепок состояния репозитория в некоторый момент времени. 

\subsection{}
С точки зрения git, файлы бывают отслеживаемые и неотслеживаемые (untracked), то есть, такие файлы, которые лежат в том же каталоге, что и git, но игнорируемые системой (временные файлы, файлы настроек IDE, логи и так далее). Отслеживаемые файлы, в свою очередь, могут быть в том же состоянии, что и в предыдущем коммите (unmodified), или содержать изменения. При этом git хранит список изменений, которые войдут в новый коммит: файлы, все изменения которых учтены, называются проиндексированными (staged). Оставшиеся файлы (то есть те, которые содержат неучтённые изменения) называются изменёнными (modified).

\end{document}