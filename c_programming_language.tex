\PassOptionsToPackage{unicode=true}{hyperref}

% print mode
\documentclass[handout]{beamer}
% presentation mode
%\documentclass{beamer}

\usetheme{Montpellier}
\usepackage[T2A]{fontenc}
\usepackage[utf8x]{inputenc}
\usepackage[russian]{babel}
\usepackage{default}

\usepackage{minted}

\usepackage{fancyvrb}

\usepackage{tikz}
\usetikzlibrary{positioning,arrows,shapes,shadows,calc}

\usepackage{graphicx}
\usepackage{color}

\tikzstyle{dir} = [
ellipse,
thick,
minimum size=1cm,
draw=red!50!black!50,
top color=white,
bottom color=red!50!black!20,
drop shadow
]

\tikzstyle{workdir} = [
ellipse,
thick,
minimum size=1cm,
draw=red!50!black!70,
top color=white,
bottom color=red!50!black!50,
drop shadow
]

\tikzstyle{file} = [
rounded rectangle,
thick,
minimum size=1cm,
draw=blue!50!black!50,
top color=white,
bottom color=blue!50!black!20,
drop shadow
]

\tikzstyle{blkdev} = [
diamond,
thick,
minimum size=1cm,
draw=yellow!50!black!50,
top color=white,
bottom color=yellow!50!black!20,
drop shadow
]

\definecolor{listing}{rgb}{0.95,0.95,0.95}

\title{Программирование на языке Си}
\date{}

\begin{document}
	
\begin{frame}
	\maketitle
\end{frame}
	
\begin{frame}[allowframebreaks]{Содержание}
	\tableofcontents
\end{frame}
	
\section{Компилятор GNU GCC}

\subsection{Компиляция}
\begin{frame}{Компилятор GNU GCC}
	\begin{itemize}
		\item{Существует множество компиляторов для языка Си. Наиболее популярным компилятором является GNU GCC}\pause
		\item{Наберите следующий пример программы на Си:}
	\end{itemize}
	\vspace*{-\baselineskip}
	\inputminted[linenos,bgcolor=listing]{C}{files/c_programming_langauge/hello.c}\pause
	\vspace*{-\baselineskip}
	\begin{itemize}
		\item{Сохраните его под именем \texttt{hello.c} и выполните команду \texttt{gcc~-o~hello~hello.c}}
	\end{itemize}
\end{frame}

\begin{frame}
	\begin{itemize}
		\item{Теперь в Вашем рабочем каталоге появился исполняемый файл \texttt{hello}. Запустите его командой \texttt{./hello}}\pause
		\item{Почти все программы на Си состоят из нескольких исходных файлов. Процесс их сборки немного сложнее, чем для программы из одного файла. Он будет рассмотрен подробно на соответствующем семинаре.}\pause
		\item{Компилятор GCC поддерживает много разнообразных опций. Например, ключи \texttt{-O0}, \texttt{-O1}, \texttt{-O2} и \texttt{-O3} задают то, какие оптимизации будет применять компилятор (0 --- без оптимизаций, применяется для отладки, 1 --- базовые оптимизации, 2 --- стабильные оптимизации, 3 --- экспериментальные оптимизации)}\pause
		\item{Часто используется опция \texttt{-g}. Она необходима для добавления к исполняемому файлу отладочной информации}
	\end{itemize}
\end{frame}

\subsection{Отладка с помощью GDB}
\begin{frame}{Отладка с помощью GDB}
	\begin{itemize}
		\item{Для того, чтобы понять, как работает программа и где в ней ошибка, часто бывает полезен \emph{отладчик}: это программа, с помощью которой можно шаг за шагом проследить за ходом работы отлаживаемой программы}\pause
		\item{Пересоберите программу \texttt{hello} с ключом \texttt{-g} и запустите отладчик: \texttt{gdb ./hello}}\pause
		\item{Добавьте точку останова --- строчку, на которой необходимо прервать выполнение программы. Для этого наберите команду \texttt{b main} (b --- это сокращение от \emph{breakpoint}, слово main в качестве аргумента говорит отладчику, что нужно остановиться перед первой инструкцией функции \texttt{main})}
	\end{itemize}
\end{frame}

\begin{frame}
	\begin{itemize}
		\item{Теперь нужно запустить программу с помощью команды \texttt{r} (многие часто используемые команды сокращены до одной буквы)}\pause
		\item{Отладчик вывел, что программа остановлена на строчке 4}\pause
		\item{Чтобы перейти к следующей строчке, наберите команду \texttt{n} (сокращение от \emph{next})}\pause
		\item{Если нажать на \texttt{Enter} (ввести пустую команду), то gdb повторит предыдущую}\pause
		\item{Если на текущей строке есть вызов функции, и Вы хотите спуститься внутрь этой функции, воспользуйтесь командой \texttt{s} (сокращение от \emph{step})}\pause
		\item{Если нужно, наоборот, выйти из функции, воспользуйтесь командой \texttt{fin}}
	\end{itemize}
\end{frame}

\begin{frame}
	\begin{itemize}
		\item{Допустим, нам нужно узнать, чему равна переменная \texttt{i}. Для этого введём команду \texttt{p i} (сокращение от \emph{print})}\pause
		\item{Выводить можно не только значения переменных, но и любых выражений, вычисление которых не изменит состояния программы (в терминологии C++11, только constexpr-выражения)}\pause
		\item{Если необходимо видеть значение выражения постоянно, замените команду \texttt{p} командой \texttt{d} (от \emph{display})}\pause
		\item{Чтобы видеть текст программы, воспользуйтесь либо командой \texttt{l}, которая выведет часть исходного текста вокруг текущей строчки, либо командой \texttt{+}, которая переведёт gdb в псевдографический режим}\pause
		\item{Если отладка данной части кода завершена, и Вы хотите перейти к следующей точке останова, воспользуйтесь командой \texttt{c} (от \emph{continue})}\pause
		\item{Для выхода служит команда \texttt{q}}
	\end{itemize}
\end{frame}

\section{Особенности языка Си}

\subsection{Структуры}
\begin{frame}{Структуры}
	\inputminted[linenos,bgcolor=listing,fontsize=\small]{C}{files/c_programming_langauge/struct.c}
\end{frame}

\subsection{Битовые поля}
\begin{frame}{Битовые поля}
	\inputminted[linenos,bgcolor=listing,fontsize=\small]{C}{files/c_programming_langauge/bitfield.c}
\end{frame}
	
\subsection{Typedef}
\begin{frame}{Typedef}
	\inputminted[linenos,bgcolor=listing,fontsize=\small]{C}{files/c_programming_langauge/typedef.c}
\end{frame}

\subsection{Указатели на функции}
\begin{frame}{Указатели на функции}
	\inputminted[linenos,bgcolor=listing,fontsize=\small]{C}{files/c_programming_langauge/funcptr.c}
\end{frame}

\subsection{Функции с переменным числом аргументов}
\begin{frame}{Функции с переменным числом аргументов}
	\inputminted[linenos,bgcolor=listing,fontsize=\small]{C}{files/c_programming_langauge/vararg.c}
\end{frame}

\subsection{Стандартные функции ввода-вывода}
\begin{frame}{Стандартные функции ввода-вывода}
	\inputminted[linenos,bgcolor=listing,fontsize=\small]{C}{files/c_programming_langauge/stdio.c}
\end{frame}

\begin{frame}{Многофункциональность функций \texttt{printf()}, \texttt{scanf()}}
	\begin{center}
		\begin{tabular}{|c|p{8cm}|}
			\hline
			Префикс & Назначение \\ \hline
			\texttt{f} & Работа с произвольным файлом (\texttt{FILE *}) вместо стандартных потоков ввода и вывода\\ \hline
			\texttt{s} & Работа со строковым буфером (небезопасно, так как выход за границу буфера не отслеживается) \\ \hline
			\texttt{sn} & Работа со строковым буфером с указанием его длины\\
			\hline
			\texttt{v*} & Всё то же самое, но с использованием \texttt{va\_list} \\ \hline
		\end{tabular}
	\end{center} \pause
	Для работы эти функции используют так называемые \emph{форматирующие} строки. Описание: {\color{blue}\url{http://en.cppreference.com/w/c/io/fprintf}}
\end{frame}

\subsection{Работа с динамической памятью}
\begin{frame}{Работа с динамической памятью}
	\inputminted[linenos,bgcolor=listing,fontsize=\small]{C}{files/c_programming_langauge/malloc.c}
\end{frame}

\subsection{Статические переменные}
\begin{frame}{Статические переменные внутри функций}
	\inputminted[linenos,bgcolor=listing,fontsize=\small]{C}{files/c_programming_langauge/static.c}
\end{frame}

\begin{frame}{Статические переменные вне функций}
	\texttt{foo.c}
	\inputminted[linenos,bgcolor=listing,fontsize=\small]{C}{files/c_programming_langauge/static_foo.c}
	\texttt{main.c}
	\inputminted[linenos,bgcolor=listing,fontsize=\small]{C}{files/c_programming_langauge/static_main.c}	
\end{frame}

\subsection{Макросы}
\begin{frame}{Макросы}
	\inputminted[linenos,bgcolor=listing,fontsize=\small]{C}{files/c_programming_langauge/macros.c}
\end{frame}

\subsection{extern, volatile, auto}
\begin{frame}{\texttt extern}
	\texttt{foo.c}
	\inputminted[linenos,bgcolor=listing,fontsize=\small]{C}{files/c_programming_langauge/extern_foo.c}
	\texttt{main.c}
	\inputminted[linenos,bgcolor=listing,fontsize=\small]{C}{files/c_programming_langauge/extern_main.c}
\end{frame}
	
\begin{frame}{\texttt volatile}
	\inputminted[linenos,bgcolor=listing,fontsize=\small]{C}{files/c_programming_langauge/volatile.c}
\end{frame}

\begin{frame}{\texttt auto}
	\begin{itemize}
		\item{\texttt{auto} в Си имеет не такой смысл, как в C++}\pause
		\item{В ранних версиях языка использовалось как антоним \texttt{extern}}\pause
		\item{Сейчас сохраняется ради обратной совместимости и не рекомендуется к использованию}
	\end{itemize}
\end{frame}

\subsection{Глобальные переменные}
\begin{frame}{Глобальные переменные}
	\begin{itemize}
		\item{Если есть несколько файлов с одной и той же глобальной переменной, то произойдёт ошибка компиляции\footnote{На самом деле, линковки}}\pause
		\item{Если несколько потоков используют одну и ту же глобальную переменную (например, \texttt{errno}), часть значений может потеряться}\pause
		\item{Решение --- статические переменные}
	\end{itemize}
\end{frame}
	
\end{document}